\documentclass[12pt,letterpaper]{article}
\usepackage{fullpage}
\usepackage[top=2cm, bottom=4.5cm, left=2.5cm, right=2.5cm]{geometry}
\usepackage{amsmath,amsthm,amsfonts,amssymb,amscd}
\usepackage{lastpage}
\usepackage{enumerate}
\usepackage{fancyhdr}
\usepackage{mathrsfs}
\usepackage{xcolor}
\usepackage{graphicx}
\usepackage{listings}
\usepackage{hyperref}

\hypersetup{%
  colorlinks=true,
  linkcolor=blue,
  linkbordercolor={0 0 1}
}
 
\renewcommand\lstlistingname{Algorithm}
\renewcommand\lstlistlistingname{Algorithms}
\def\lstlistingautorefname{Alg.}

\lstdefinestyle{Python}{
    language        = Python,
    frame           = lines, 
    basicstyle      = \footnotesize,
    keywordstyle    = \color{blue},
    stringstyle     = \color{green},
    commentstyle    = \color{red}\ttfamily
}

\setlength{\parindent}{0.0in}
\setlength{\parskip}{0.05in}

% Edit these as appropriate
\newcommand\course{BEE 4750}
\newcommand\hwnumber{1}                  % <-- homework number
\newcommand\NetIDa{Student Name}           % <-- Name of person #1
%\newcommand\NetIDb{netid12038}           % <-- Name of person #2 (Comment this line out for problem sets)

\pagestyle{fancyplain}
\headheight 35pt
\lhead{\NetIDa}
%\lhead{\NetIDb\\\NetIDb}                 % <-- Comment this line out for problem sets (make sure you are person #1)
\chead{\textbf{\Large Homework \hwnumber}}
\rhead{\course \\ \today}
\lfoot{}
\cfoot{}
\rfoot{\small\thepage}
\headsep 1.5em

\begin{document}

\section{Problem Description}
%\section*{Problem Description} %use this version to have a section heading without numbering.

Describe the problem here
You can create bullets below to list things if you like. (enumerate is a numbered list, itemize is just bullets)
\begin{enumerate}
  \item
   Problem 1 part 1 answer here.
  \item
    Problem 1 part 2 answer here.
\end{enumerate}

 These are a few examples that you are welcome to use to set up your formulations. Also there are lots of resources online. For example, I wanted to talk about the ampersand without having LaTeX interpret it as a command, so I googled "latex verbatim" and found an example to cut and paste, as well as the commands above that loaded the package I needed. (the one called verbatim and the line below defining the environment - see lines 14 and 15 in the source)

\section{A simple layout} 

$$\begin{align*}
    & \max z = 3x_1 + 4x_2  \\
    &\text{subject to} &&\\
    & 3x_1 + 6x_2  &\leq& 27\\
    & x_2 &\geq& 2\\
    & 3x_1 + x_2 &\leq& 19 \\
    & x_1, x_2 \geq 0 \quad \text{and integer}
\end{align*}$$

\section{ Alignment}
You have to play with the
\begin{verbatim} & character \end{verbatim} 
to get the alignment how you want it. Here is another example:

$$
\begin{align}
   &\max & Z = X_1 + X_2 \\
   &\text{s.t.} &  5X_1 + 10X_2 \leq 40 \\
   && 6X_1 + 3X_2 \leq 18\\
   && X_1, X_2 \geq 0 
\end{align}
$$

\section{Scalable brackets and braces}
Note that in the code below, (see source) the use of
\begin{verbatim}\left \left{ \end{verbatim} 
tells latex to scale the bracket to the size of what's inside. I think you also have to end with the corresponding 
\begin{verbatim}\right \right} \end{verbatim} 
to avoid an error. 

$$\min \left \{ \mathbb{E}\left[ \sum_{t=1}^T S_g(u_g^t - u_g^{t-1})  + c_t(\cdot) \right] \right \}$$


\section{Case statements for choices}
Below is an example of a "case" for writing if/then type of things. 


$$C_j Y_j + B_j \sum_i W_{ij}$$
$$ \[ Y_j = 
\begin{cases}
0 & \text{if} \quad \sum_i W_{ij} = 0 \\
1 & \text{if} \quad \sum_i W_{ij} > 0 
\end{cases}
\] $$
\section{Inserting Images}

    Here is an example of how you can insert a figure.
    \begin{figure}[!ht]
    \centering
    \includegraphics[width=0.3\linewidth]{BubbleImage.png}
    \caption{Just a test image.}
    \end{figure}

From L. Sendelback to class on Slack (Fall 2018):
\begin{verbatim}Inserting the image in LaTeX is easy if you add the command
\includegraphics[width=\textwidth]{ImageNameHere}
where you want the figure as long as you
\usepackage{graphicx}
at the very top of the document and you upload the image in Overleaf\end{verbatim}

\end{document}
